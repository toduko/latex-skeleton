\documentclass[11pt, a5paper]{report}

\usepackage[utf8]{inputenc}
\usepackage[T2A]{fontenc}
\usepackage[english, bulgarian]{babel}
\usepackage{amssymb}
\usepackage{hyperref, fancyhdr, lastpage, fancyvrb, tcolorbox, titlesec}
\usepackage{array, tabularx, colortbl}
\usepackage{tikz}
\usepackage{venndiagram}
\usepackage{amsthm}
\usepackage{amsmath,physics}
\usepackage{mathtools}
\usepackage[a5paper, left=0.50in, right=0.50in, top=1.0in, bottom=1.0in]{geometry}

\ExplSyntaxOn
\NewDocumentCommand{\opair}{m}
 {
  \langle\mspace{2mu}
  \clist_set:Nn \l_tmpa_clist { #1 }
  \clist_use:Nn \l_tmpa_clist {,\mspace{3mu plus 1mu minus 1mu}\allowbreak}
  \mspace{2mu}\rangle
}
\ExplSyntaxOff

\hypersetup{
    colorlinks=true,
    linktoc=all,
    linkcolor=blue
}

\newtheorem{definition}{Дефиниция}[section]
\newtheorem{theorem}[definition]{Теорема}
\newtheorem{claim}[definition]{Твърдение}
\newtheorem{axiom}[definition]{Аксиома}
\newtheorem{lemma}[definition]{Лема}
\newtheorem{corollary}[definition]{Следствие}
\theoremstyle{remark}
\newtheorem*{remark}{Забележка}

\pagestyle{fancy}

\lhead{\leftmark}
\rhead{}

\setlength\parindent{0pt}

\title{Скелет на документ}
\author{Тодор Дуков}
\date{}

\begin{document}
\maketitle
\tableofcontents
\chapter{Примерна глава}
\section{Примерна секция}

Нека като за начало да дадем една аксиома.

\begin{axiom}[Логическа аксиома]
    Съществува множество т.е.

    \centerline{$( \exists x ) ( x = x)$}
\end{axiom}

Сега нека дадем и една дефиниция.

\begin{definition}[Наредена двойка]
    Ако x и y са множества, то наредена двойка на x и y ще наричаме множеството $\{\{x\}, \{x, y\}\}$ и ще го бележим с $\opair{x, y}$.
\end{definition}

\begin{claim}
    $( \forall n \in \mathbb{N} ) ( \displaystyle\sum_{i=1}^{n}i = \frac{n(n+1)}{2} )$
\end{claim}

\begin{proof}
    Ще докажем твърдението с индукция по $n \in \mathbb{N}$.

    База: При $n = 0$ се получава, че $0 = \frac{0(0+1)}{2}$ \checkmark

    ИС: $\displaystyle\sum_{i=1}^{n+1}i = \displaystyle\sum_{i=1}^{n}i + (n + 1) \overset{\text{ИП}}{=} \frac{n(n+1)}{2} + (n + 1) = \frac{n(n+1) + 2(n+1)}{2} = \frac{(n+1)(n+2)}{2}$
\end{proof}

\begin{lemma}[за покачването]
    Езикът е $L$ регулярен.
    Тогава $(\exists p \in \mathbb{N} \: \forall \alpha \in \Sigma^* \: \exists x, y, z \in \Sigma^*, xyz = \alpha, |xy| \leq p, |y| \geq 1 \: \forall i \in \mathbb{N}) (xy^iz \in L)$.
\end{lemma}

\begin{proof}
    Езикът $L$ е регулярен.

    Следователно съществува ДКА
    $\mathcal{A} = \opair{Q, \Sigma, s, \delta, F}$,
    такъв че $\mathcal{L}(\mathcal{A}) = L$.

    Нека $p = |Q|$ и нека $q_1, \dots, q_p$ са състоянията от $Q$.
    Нека $\alpha \in L$ е такава, че $|\alpha| = n$, където $n \geq p$.

    Ще разбием $\alpha$ на $\alpha_1, \dots, \alpha_n \in \Sigma$ (т.е. $\alpha = \alpha_1\dots\alpha_n$).
    Знаем, че съществуват $q_{i_0} = s, \dots, q_{i_n} \in Q$, такива че
    $(\forall j \in \{1, \dots, n\}) (\delta(q_{i_{j-1}}, \alpha_j) = q_{i_j})$.

    Нека разгледаме думата $\alpha_1 \dots \alpha_p$.
    За нея знаем, че по време на четенето на думата
    автоматът минава през $p + 1$ състояния.
    Следователно по принципът на Дирихле съществуват
    $t_1, t_2 \in \{1, \dots, p\}$, където $t_1 < t_2$,
    такива че $q_{i_{t_1}} = q_{i_{t_2}}$.

    Нека
    $x = \alpha_1 \dots \alpha_{t_1}$,
    $y = \alpha_{t_1 + 1} \dots \alpha_{t_2}$,
    $z = \alpha_{t_2 + 1} \dots \alpha_n$.
    Сигурни сме, че $|xy| \leq p$, защото $t_2 \leq p$ и че $|y| \geq 1$ понеже $t_1 \neq t_2$.
    Знаем, че $\delta^*(q_{i_{t_1}}, y) = q_{i_{t_2}}$.

    \begin{claim}
        $(\forall i \in \mathbb{N}) (\delta^*(q_{i_{t_1}}, y^i) = q_{i_{t_2}})$.
    \end{claim}

    \begin{proof}
        Ще докажем твърдението с индукция по $i \in \mathbb{N}$.

        База: $\delta^*(q_{i_{t_1}}, \epsilon) = q_{i_{t_1}} = q_{i_{t_2}}$ \checkmark

        ИС: $\delta^*(q_{i_{t_1}}, y^{i+1}) \overset{\text{Деф}}{=} \delta(\delta^*(q_{i_{t_1}}, y^i), y) \overset{\text{ИП}}{=} \delta(q_{i_{t_2}}, y) = \delta(q_{i_{t_1}}, y) = q_{i_{t_2}}$
    \end{proof}

    Знаем, че $\alpha = xyz \in L$.
    Тъй като $xyz = \alpha$, имаме че $\delta^*(s, xyz) \in F$.
    От тук следва, че $\delta^*(\delta^*(s, xy), z) \in F$,
    следователно $\delta^*(\delta^*(\delta^*(s, x), y), z) \in F$.
    $\delta^*(s, x) = q_{i_{t_1}} \Rightarrow \delta^*(\delta^*(q_{i_{t_1}}, y), z) \in F$.
    От доказаното твърдение имаме, че $(\forall i \in \mathbb{N}) (\delta^*(q_{i_{t_1}}, y^i) = q_{i_{t_2}} = q_{i_{t_1}})$.
    Освен това знаем, че $\delta^*(q_{i_{t_2}}, z) \in F$.
    Следователно
    $(\forall i \in \mathbb{N}) (\delta^*(\delta^*(q_{i_{t_1}}, y^i), z) \in F)$.
    От тук можем да заключим, че
    $(\forall i \in \mathbb{N}) (\delta^*(\delta^*(\delta^*(s, x), y^i), z) \in F)$
    и вървейки в обратната посока (обединяването на всички $\delta^*$) получаваме, че
    $(\forall i \in \mathbb{N}) (\delta^*(s, xy^iz) \in F)$,
    с което доказахме лемата.

\end{proof}

\begin{remark}[Примерна забележка]
    $( \forall x ) (\varnothing \subseteq x)$
\end{remark}

\chapter{Друга примерна глава}

\section{Примерна секция}
Това е просто случаен текст, който си губиш времето да четеш.
\section{Друга примерна секция}
Това е просто случаен текст, който си губиш времето да четеш.
\end{document}